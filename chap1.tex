\chapter{Mori Culture Library}

\section{Digital library}

\subsection{Introduction}
Digital library refers to the library that uses digital technology to store, organize, retrieve, transmit, utilize and protect document resources. Different from traditional physical libraries, digital libraries are mainly dominated by literature resources in electronic form, which can be retrieved and accessed through computer networks. Digital library is the product of the information age. It uses the digital way to manage the document resources, which makes it more convenient for people to obtain and use the information resources, and also provides a broader space for academic research and education.

Digital library is of great significance to academic research and education. It provides more extensive and deeper information resources for education and research, and provides a more convenient and efficient learning, research and teaching platform for scholars, researchers and students. At the same time, digital library also provides a wider range of cultural and knowledge resources for the society, and promotes cultural exchange and knowledge innovation.

\subsection{Main Features}
\subsubsection{Electronic resources}
The resources of digital library are mostly stored in electronic form, including electronic books, digital newspapers, journals, dissertations, patents, etc., as well as audio, video, pictures and other forms of digital resources.

\subsubsection{Resource sharing}
Digital library allows users to use the network to access library resources anytime and anywhere, not limited by time and place, while digital libraries can also share resources, to provide readers with a wider range of information resources.

\subsubsection{Search function}
Digital library has powerful retrieval function, users can find the information resources they need by various retrieval methods, such as keyword search, classification search, author search and so on.

\subsubsection{Knowledge management}
Using advanced knowledge management technology, digital library can effectively organize and manage literature resources, so that readers can obtain information more conveniently.

\subsubsection{Globally}
The resources of digital library cover a wide area and are not restricted by region. Users can obtain the literature resources from all over the world at any time through the network.

\subsection{Content}
In general, digital library includes literature resources, metadata, search tools, digital library systems, services and support, and rights management. These contents together constitute the basic elements of digital library, providing users with convenient, fast and efficient information resources services. 
 
\subsubsection{Literature resources}
The literature resources collected by the digital library include electronic books, digital newspapers, journals, dissertations, patents, etc., as well as audio, video, pictures and other forms of digital resources. These resources can be accessed online or downloaded on the digital library website.

\subsubsection{Metadata}
Metadata is the data that describes the literature resources. It includes the basic information of the literature (such as title, author, publisher, publication time, etc.) as well as the standardized classification, index, and label. The role of metadata is to provide users with more convenient and accurate retrieval services.

\subsubsection{Search tools}
Digital libraries provide various search tools, such as keyword search, classification search, and author search, to help users quickly find the required literature resources. The efficiency and accuracy of retrieval tools have a significant impact on the user experience of digital libraries.

\subsubsection{Digital library system}
Digital library system refers to the software system that supports the operation of digital library, including digital processing system, resource management system, retrieval system, user interface, etc. The development and maintenance of digital library system is very important for the stability and sustainable development of digital library.

\subsubsection{Services and Support}
Digital libraries provide a variety of services and support, such as reader consultation, academic paper writing guidance, and bibliography recommendation, to help users make better use of digital library resources. The digital library also regularly organizes trainings, seminars and other events to promote academic exchanges and knowledge sharing.

\subsubsection{Copyright Management}
Digital libraries need to comply with copyright laws and related regulations to protect the copyright and interests of authors and publishers. Digital libraries need to develop reasonable rights management strategies, including licensing agreements, digital rights protection technologies, copyright supervision and other measures, to ensure the legitimacy and sustainable development of digital libraries
 
\subsection{Usage}
In order to give full play to the role of digital library, enterprises should ensure the effective protection of information retrieval, knowledge management, training and education after the establishment of digital library system. Enterprises also need to be equipped with special digital library administrators, responsible for the maintenance and update of digital library, to ensure that the resources and functions of digital library always keep the latest, the most comprehensive, the most valuable.

\subsection{Cost}
In general, most of the cost of digital library lies in the establishment and maintenance. However, the advantage of digital library is that it can greatly save the storage space and maintenance cost of traditional library, and digital library can provide richer and more convenient resources and services to bring better experience for users.
 
\section{Project analysis - Waikatohea}
\subsection{Maori Culture}
Maori culture refers to the culture of the Maori people, the indigenous people of New Zealand. The Maori are the indigenous people of New Zealand. They have a rich history and cultural tradition that includes a unique language, art, music, dance, traditional medicine, architecture and food.

Maori culture attaches great importance to harmonious coexistence with nature, and believes that various natural phenomena have their own soul and meaning, which should be respected and protected. Maori culture also attaches great importance to family and community ties, and emphasizes teamwork and mutual assistance.

In order to promote Maori culture around the world, our project as a digital library platform will focus on presenting Maori culture based on the Maori language and supplemented by English.

\subsection{Project Objects}
The content of our project - Maori Cultural Digital Library is very rich and diverse. It mainly includes the following aspects, some of which can be made into electronic materials.

\subsubsection{Maori language and culture Materials}
The Maori Culture Digital Library collects and preserves a large number of Maori language and culture materials, including Maori language dictionaries, grammar and pronunciation tutorials, Maori cultural traditional stories and myths, Maori cultural history, Maori cultural music and dance, etc.

\subsubsection{Maori Arts and Crafts}
The Digital Library of Maori Culture also collects and displays many Maori arts and crafts, such as traditional Maori bone carving, wood carving and jade carving art, Maori fabrics and woven goods, Maori jewelry and ornaments, etc.

\subsubsection{Maori History and Cultural Geography}
The Maori Culture Digital Library also provides information on the history and cultural geography of the Maori people, providing insight into their origins and historical development, and understanding Maori cultural and social practices throughout New Zealand.

\subsection{Project Content}
The content format of digital library is mainly based on the following formats:

\subsubsection{Digital Books}
The Maori Culture Digital Library provides a large number of digital books, including Maori culture history, Maori culture and art, Maori culture traditional stories and myths, etc.

\subsubsection{Video and audio files}
The Maori Culture Digital Library offers a number of video and audio files that allow people to enjoy music and dance performances of Maori culture and hear the Maori language spoken and spoken.

\subsubsection{Image and text documents}
The Digital Library of Maori Culture also provides a number of image and text documents, including photographs of Maori cultural artworks, texts of Maori cultural traditional stories and myths, etc.

\subsubsection{Interactive Applications}
The Maori Culture Digital Library also provides many interactive applications, such as the Maori language learning application, the 3D browsing application of Maori cultural artworks, etc., so that people can understand and experience the Maori culture more deeply.

\subsection{Project Direction}
Our project can be made into a digital library of Maori culture, which aims to collect, preserve, display and disseminate the Maori cultural heritage online platform.

Digital Collection of Maori Cultural Heritage: The Maori Cultural Digital Library collects important Maori heritage such as traditional culture, history, art and language through digital technology, forming a digital cultural database, so that more people can easily access and understand Maori culture.

Promoting the inheritance and development of Maori culture: The Digital Library of Maori Culture is not only a platform for the collection and preservation of cultural heritage, but also an important tool for promoting the inheritance and development of Maori culture. Through digital display and dissemination, more people can learn and understand Maori culture, and help Maori themselves better protect and pass on their own culture.

Providing education and research resources: The Digital Library of Maori Culture is not only a showcase, it also provides a wealth of education and research resources, so that students, teachers and researchers can understand and study Maori culture more deeply, and promote the development of Maori cultural research.

Collaboration with other cultural digital libraries: Maori cultural digital libraries can collaborate with other cultural digital libraries to share cultural resources and experiences and enhance cultural exchange and understanding. In this way, more people can know about Maori culture, and at the same time, Maori can better understand and respect other cultures.

Protecting and inheriting the sustainability of Maori Culture: The Maori Cultural Digital Library is an important tool for protecting and inheriting Maori culture, which needs to be constantly updated and improved to ensure the sustainability of Maori cultural heritage. At the same time, it is also necessary to actively participate in community cultural activities to promote the development and inheritance of Maori culture. 
 
 
