\chapter{Data collection}

\section*{Introduction}
\lettrine[lines=2]{L}\\ onrix's business scenario requires the scanning of the road by means of a camera placed in a moving vehicle at about 50km/h to produce measurements of the target in both horizontal and vertical directions.

\paragraph*{}
In this first chapter, we will start with some basic definitions and then focus on how ZED cameras can collect data at a variety of speeds, angles and altitudes.

\section{ZED camera}
 
AAAAAAAAAAAAAAAAAABBBBBBBBBBBBBB
\section{Camera optical concepts}

\subsection{Polariser}

 \subsection{Shutter mode}
 
\subsection{Automatic exposure}
 
\subsection{Focal length lens}

 
\section{Basic assumptions for data collection}

 
\section{Data collection environment}
 

\begin{figure}[htbp]
  \centerline{\includegraphics[width=200pt]{images/UoW.jpg}}
  \caption{Screenshot of Zed Explorer }
  \label{zede}
\end{figure} 

You can find settings to turn off auto-exposure and a range of other parameters.
They are only valid before each recording, and these values can no longer be changed in the finished SVO file.

The resolution and FPS can be adjusted in the multi selection in the top left corner, but only before the svo recording starts. 
If this option only has a very low resolution, it is because the usb cable is faulty and cannot support high speed transfers.

\section{All collected datasets}

 
\subsection{Feb. 10 at Hillcrest}
 
SN31380347\_10-54-05.svo (this file is from Feb. 15 to fix the frame lost issue).

\subsection{Feb. 15 at Hillcrest}

This dataset tried to fix the issue of previous one. 

\subsection{Feb. 10 at MacDonald}

This data was collected for very slow speeds (less than 20km/h) with obvious problems on the ground.

\subsection{Feb. 20 at Lonrix Office}

This dataset was collected on the ground in the car park outside the Lonrix office and the distances were marked using a measuring tape.

\subsection{Feb. 21 near Lonrix Office}

This dataset collected vertical poles and traffic signs within different ranges inlcuding 5m and 10m. 

\subsection{Mar. 02 near the University of Waikato}

This dataset collected some vertical traffic signs that have been measured in person to obtain ground truth.

\section{Overview of the collection}

Overall, care should be taken when collecting data to make adjustments to optical parameters. 

\subsection{Camera positions and angles}

\renewcommand{\labelitemi}{$\blacksquare$}
\begin{itemize}     
    \item As the accurate range of the camera is limited, the closer the camera is positioned to the target object the better, but as the camera requires a USB cable connection, be sure to note that a long USB cable is required.

    \item Whether on the top or on the bonnet of the car, the difference is not that great.
    
    \item The angle of the camera has an effect, especially on more distant vertical targets.
    
    \item It is important to keep the camera angle as perpendicular to the ground as possible, otherwise this will result in more long-tailed vertical pole point clouds.
\end{itemize}

\subsection{USB3.0}

ZED cameras use USB 3.0 in order to transfer data in high resolution, but the camera interface is USB-C. 

\subsection{ZED SDK and GPU}

The ZED SDK only supports Windows and Ubuntu and does not support macOS. 

\section*{Conclusion}
Collecting data with the ZED is relatively easy, but hardware preparation especially USB cables, GPU machines, or the right model of camera itself can make collection difficult.

