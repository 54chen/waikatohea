\chapter{Abstract}
\chaptermark{Abstract}

Regarding Whakatōhea's initiative to create an ebook, we have successfully developed both front-end and back-end prototypes, and conducted in-depth research on AI generation techniques. The front-end prototype provides a user-friendly interface for users to interact with the ebook, while the back-end prototype focuses on utilizing AI-generated content. Through our research, we have explored various AI tools and methodologies to facilitate the generation of ebook materials, including text, images, and audio. Furthermore, we have investigated the state-of-the-art AI models, such as GPT-3 and DALL·E, to leverage their capabilities in generating meaningful and contextually relevant content. We have also considered the challenges associated with incorporating Māori language support and sought out suitable solutions, such as utilizing Microsoft's Māori language translation services. Overall, our comprehensive research and prototype development efforts have laid a solid foundation for Whakatōhea's ebook project, enabling efficient and effective content creation processes.

\chapter{Introduction}
\chaptermark{Introduction}

Whakatōhea, an indigenous Māori iwi, possesses a collection of traditional books that hold significant cultural value. In order to preserve and expand the accessibility of these valuable resources, Whakatōhea has embarked on an ebook initiative. Our research team has successfully developed comprehensive front-end and back-end prototypes to support this endeavor. The front-end prototype offers a user-friendly interface that allows seamless interaction with the ebook, while the back-end prototype focuses on harnessing the power of AI-generated content.

Our research efforts have delved deeply into the realm of AI generation techniques, exploring various methodologies for generating text, images, and audio content. Through rigorous experimentation, we have examined state-of-the-art AI models, including GPT-3 and DALL·E, to leverage their capabilities in producing meaningful and culturally relevant content for the ebooks. We have also given careful consideration to the unique challenges associated with incorporating Māori language support, and have explored solutions such as utilizing Microsoft's Māori language translation services.

By completing these prototypes and conducting extensive research on AI generation technologies, we have laid a solid foundation for Whakatōhea's ebook project. The outcomes of our efforts will provide Whakatōhea with efficient and effective methods for transforming traditional books into digital formats, thereby ensuring the preservation, dissemination, and enhanced accessibility of their cultural heritage.

The Whakatōhea, a Māori tribe situated in the Eastern Bay region on the east coast of New Zealand's North Island, has encountered challenges such as land loss in the past. Their commitment lies in the preservation and upkeep of their traditional values and way of life \autocite{Whakatoh25:online}. Our objective in establishing the digital library is to safeguard the tribe's history, traditional knowledge, and cultural heritage, ensuring these valuable resources are not lost or forgotten over time. By means of the digital library, the younger generation will have an enhanced understanding and learning experience of their traditional culture and history.

The digital library will incorporate Māori language learning materials into textbooks and create an environment for Māori language learning, thereby offering a curriculum designed for young individuals aged 13-18 years old. Key requirements for this digital library encompass the conversion of eBooks, inputting them into the backend database, and presenting them on the frontend to enable functions like deletion, modification, and search. The eBooks will be classified based on age groups and relevant historical topics, facilitating the search process.
