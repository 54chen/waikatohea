\chapter{In-depth study of 3D Registration}

\section{Introduction}
\lettrine[lines=2]{I}\\n order to obtain highly accurate depth images over a large range on a low accuracy device, it becomes possible to stitch multiple frames from the camera by aligning the point cloud.
The high accuracy over a short range is extended to the accuracy over a long range by algorithms.

\subsection{3D reconstruction and Depth estimation}
 
It requires a device that supports time-of-flight (TOF) lasers, microwaves or 3D ultrasound. 

\subsection{Depth and Point clouds}

The depth information can be used to generate point clouds, or to augment existing point clouds with additional depth information. 

\subsection{Point clouds and Registration}

Point cloud registration is often used to combine or merge multiple point clouds into a single, comprehensive model of a scene or object. 

\subsection{ICP}

Iterative closest point (ICP) is a widely used algorithm in 3D computer vision and robotics for aligning and registering two or more point clouds or 3D models. 

\subsection{SVO file format}

SVO is the recorded video file format from the ZED camera. 

\section{Direct performance of the ZED camera without any processing}

For fast svo file processing, a toolbox project was implemented. In this project, the most important data is the depth data that can be dump from SVO file and stored as npz file.

\begin{lstlisting}

  distance = math.sqrt((X1 - X2) * (X1 - X2) +
                        (Y1 - Y2) * (Y1- Y2) +
                       (Z1 - Z2) * (Z1 - Z2))
                      
\end{lstlisting}

\subsection{Parameters of ZED}

Because ZED uses both Neural models and stereo methods to predict the depth value, there are a lot of parameters. 


\begin{lstlisting}

  depth_mode = NEURAL 
  depth_minimum_distance = 0.01 m 
  depth_maximum_distance = 30 m 

\end{lstlisting}

These parameters can be modified after the SVO has been generated.

\subsection{Best working distance}

The camera, which is ZED 2i 2.1mm with polarizer, works well in the outdoor environment when the depth value is in the near range (0m-7.5m).

\subsection{Resolutions}

The higher resolution will provide more details but the FPS will be reduced. 

\subsection{The shortage of ZED}

 
In Lonrix scenario, the most common objective is the traffic sign, which is a white or blue metal panel. (Fig.~\ref{fig2c}.)

\begin{figure}[htbp]
  \centerline{\includegraphics[width=250pt]{images/UoW.jpg}}
  \caption{Different surfaces and ranges}
  \label{fig2c}
\end{figure}


It is not too bad when close enough in the actual test results.

\subsection{Results}

There are a lot of highly accurate measurements that results in different resolutions and speeds, which is shown in Fig.~\ref{fig2d}. 
 
\begin{figure}[htbp]
  \centerline{\includegraphics[width=250pt]{images/UoW.jpg}}
  \caption{Results in different ranges}
  \label{fig2d}
\end{figure}

Fig.~\ref{fig2d2} shows an example of the different values for the same white line.
\begin{figure}[htbp]
  \centerline{\includegraphics[width=200pt]{images/UoW.jpg}}
  \caption{2.86m vs 4.08m}
  \label{fig2d2}
\end{figure}

A sufficiently dense and accurate point cloud is the basis for accurate measurements.

\section{Optimising point clouds}

There are quite a few issues if the range is beyond 7m. 

\begin{figure}[htbp]
  \centerline{\includegraphics[width=200pt]{images/UoW.jpg}}
  \caption{HD2k/50km/h, open3d image, 14m}
  \label{fig30}
\end{figure}

\subsection{ICP registration}

Fig.~\ref{fig3a} shows the single frame that only keeps 7m.

\begin{figure}[htbp]
  \centerline{\includegraphics[width=200pt]{images/UoW.jpg}}
  \caption{HD720 50km/h Frame No. 38 }
  \label{fig3a}
\end{figure}

 With the camera pose data, a pose matrix can be generated, which makes it easier to convert from local to world coordinates

\subsection{More on 3d reconstruction}

Because of the ICP registration, it is possible to get entire 3d model in an area, so that it is possible to get the measurements at various angles.

\section{Code}

The toolbox project can deal with the SVO files and do the measurements. 

\subsection{Dump files from SVO file}

This file is used for dumping the SVO file into there kinds of file: JPG, PCD and NPZ.

\begin{lstlisting}

  mkdir -p example_files/UNI-HD720/
  python runtools.py -f ~/Downloads/mar02/HD720_SN31380347\_14-31-25.svo -s example_files/UNI-HD720/ -g 1

\end{lstlisting}

The parameters represent the address of the SVO file, the directory to which it is exported, and the number of frames between each export.

\subsection{Do frame measurement showing as single color image}

\begin{lstlisting}

python runtools.py -t 3 -f example_files/UNI-HD720/ -frame 730 -show 1

\end{lstlisting}

The result will display an image, and the selected distance can be shown by clicking with a mouse. 

\subsection{Do frame measurement showing as left depth plus image mode}

\begin{lstlisting}

python runtools.py -t 3 -f example_files/UNI-HD720/ -frame 730 -show 2

\end{lstlisting}

\subsection{Do frame measurement under the YOLO result}

This command requires YOLO installation in advance.

\begin{lstlisting}

  python runtools.py -t 3 -f example_files/UNI-HD720/ -frame 730 -show 3

\end{lstlisting}

\subsection{Run colored ICP}

This tool will remove the pavement and the long tails of the pole.

\begin{lstlisting}

python runtools.py -t 4 -f example_files/UNI-HD720/ -frame 730 -rp 1 -rl 1

\end{lstlisting}

\begin{figure}[!ht]
  \centerline{\includegraphics[width=200pt]{images/UoW.jpg}}
  \caption{Screenshot of colored ICP }
  \label{fig311}
\end{figure}

\subsection{Run manual registration then run a micro distance ICP}

\begin{lstlisting}

python runtools.py -t 4 -f example_files/UNI-HD720/ -frame 730 -a 2

\end{lstlisting}

This tool will not remove the pavement and the long tails of the pole. 

\begin{figure}[htbp]
  \centerline{\includegraphics[width=200pt]{images/UoW.jpg}}
  \caption{Screenshot of manual ICP }
  \label{fig312}
\end{figure}

\subsection{Run RANSAC registration}

\begin{lstlisting}

python runtools.py -t 4 -f example_files/UNI-HD720/ -frame 730 -a 3

\end{lstlisting}

 This tool will not remove the pavement and the long tails of the pole. 

\begin{figure}[htbp]
  \centerline{\includegraphics[width=200pt]{images/UoW.jpg}}
  \caption{Screenshot of RANSAC }
  \label{fig313}
\end{figure} 


\section{Conclution and future work}

The ZED camera provides a reliable and accurate 3D sensing solution with high-quality depth sensing and stereo imaging capabilities in specific range.

\subsection{Future work}

There is some work that needs to be done in the future to ensure this camera is used in online products.

\subsection{Todo List}

