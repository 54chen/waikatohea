\chapter{Discussion}
\chaptermark{Discussion}

Through the process of this project, we have learned the entire design thinking process, starting from a prototype and continuously engaging in communication with the clients to obtain a practical prototype that effectively solves the problem. Additionally, during the communication process with Danny, we have applied customer management techniques, as he is someone who is easy to communicate with. It is important to fully explore the challenges encountered and uncover potential difficulties.

Throughout the project, we have embraced the iterative nature of design thinking, continually seeking feedback and making improvements to ensure that our solutions meet the needs and expectations of our clients. This feedback-driven and iterative approach has been crucial in obtaining a tangible prototype that effectively addresses the problem.

Furthermore, in our interactions with Danny, we have employed customer management techniques such as active listening, understanding requirements, and providing customized solutions. Danny has proven to be a communicative individual, and we aim to fully explore any challenges encountered, engaging in collaborative discussions to seek innovative solutions. This collaborative and cooperative process not only enhances our understanding of customer needs but also facilitates successful project implementation and increased customer satisfaction.

Based on the feedback from the previous discussion, we initially had a prototype with only text input for the search function. However, considering the specific needs of some users and the functionality of search, we added a voice input feature for searching. We also implemented a pop-up window for converting voice to text. Additionally, we lacked details on accessing ebooks in our original model, so we added a download ebooks feature in the download list for users to view and download ebooks at any time. Furthermore, we recognized the need for user interaction with the digital library and to achieve our goal of promoting indigenous culture. Hence, we included community interaction and quizzes to enhance user engagement.

It's a shame that this whole thing went too well, it would have been perfect if there was real conflict and conflict management related content.

We have currently achieved the basic integration of the frontend and backend models, which can adequately fulfill the usage and development of a digital library platform. In the future, there may be design enhancements to the user interface to engage the interest of young users. However, our initial goal for creating the digital library was to target teenagers, and subsequent improvements may be best suited for the general public. This will facilitate the increasing number of people learning about tribal culture.


